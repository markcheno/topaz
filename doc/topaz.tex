%% LyX 1.1 created this file.  For more info, see http://www.lyx.org/.
%% Do not edit unless you really know what you are doing.
\documentclass[english]{article}
\usepackage[T1]{fontenc}
\usepackage[latin1]{inputenc}
\usepackage{fancyhdr}
\pagestyle{fancy}
\usepackage{babel}

\makeatletter

%%%%%%%%%%%%%%%%%%%%%%%%%%%%%% LyX specific LaTeX commands.
\providecommand{\LyX}{L\kern-.1667em\lower.25em\hbox{Y}\kern-.125emX\@}

%%%%%%%%%%%%%%%%%%%%%%%%%%%%%% Textclass specific LaTeX commands.
 \newenvironment{lyxlist}[1]
   {\begin{list}{}
     {\settowidth{\labelwidth}{#1}
      \setlength{\leftmargin}{\labelwidth}
      \addtolength{\leftmargin}{\labelsep}
      \renewcommand{\makelabel}[1]{##1\hfil}}}
   {\end{list}}
 \newenvironment{lyxcode}
   {\begin{list}{}{
     \setlength{\rightmargin}{\leftmargin}
     \raggedright
     \setlength{\itemsep}{0pt}
     \setlength{\parsep}{0pt}
     \normalfont\ttfamily}%
    \item[]}
   {\end{list}}

\makeatother
\begin{document}

\title{The Topaz Programming Language}


\author{Mark Chenoweth}

\maketitle
\tableofcontents{}


\section{Introduction}


\subsection{Language Features}

Topaz is a powerful, light-weight, efficient programming language
specifically designed for the PalmOS Computing Platform. It is dynamically
typed, interpreted from byte-code, has automatic memory management,
object oriented features, access to native PalmOS gui widgets and
more.

\medskip{}
{\raggedright Topaz also Features:\par}

\begin{itemize}
\item A simple, short, and flexible syntax 
\item A simple Class/Object model for object oriented programming 
\item Built in floating point math and basic functions (sin,cos,tan,etc...) 
\item Built in graphics functions (drawpoint,drawline,drawrect,etc...) 
\item Built in native user interface classes (Button,Field,CheckBox,etc...) 
\item Automatic type conversion 
\item Easy string handling 
\item Integrated editor/compiler/interpreter
\item A single, small (<70k) .prc file with no extra libraries needed 
\item Good documentation
\end{itemize}
{\noindent \raggedright Topaz is NOT:\par}

\begin{itemize}
\item Meant to create complex multi-form applications 
\item Meant to replace more mature programming languages
\end{itemize}

\subsection{Quick Start}

To create and run the canonical {}``hello, world'' program:

\begin{enumerate}
\item Install topaz.prc into your Palm device running PalmOS V3.0 or greater
and launch Topaz.
\item Press the {}``New'' button on the Main Form to create a new program
and display the Editor Form.
\item Enter this program: 

\begin{lyxcode}
\#~Hello~world~example

println('hello,~world')
\end{lyxcode}
\item To execute the program, press the {}``Run'' button. The Output Form
will appear, and {}``hello, world'' should be printed on it. If
there are any errors, an error dialog box will notify you.
\item After the program completes executing, the Output Form will continue
to be displayed. You may return to either the Editor or the Main Form
by selecting the Menu icon, and then Goto|Editor or Goto|Main.
\item Basic program structure is as follows: 

\begin{itemize}
\item The main form will always display the first line of a program as the
program title, so programs generally start with a one line descriptive
comment followed by one or more statements.
\item Comments begin with a \texttt{\#} and continue until the end of the
line.
\item Follow the initial comment with one or more statements.
\end{itemize}
\end{enumerate}
For a more complicated example, try this:

\begin{lyxcode}
\#~Scribble~Demo~

btn~=~Button.new~

btn.setXY(10,125)~

btn.setSize(40,20)~

btn.setLabel('Clear')~

btn.show

while~true~

~~e=event(100);~y=peny~

~~if~e==btn.id~then~

~~~~eraserect(0,0,160,120)~

~~elsif~y~<~115~

~~~~drawrect(penx,y,3,3)~

~~end~

end
\end{lyxcode}

\section{Language Syntax}


\subsection{Program Structure}

A Topaz program consists of a series of statements, interspersed with
any function and/or class definitions. Statement, function, and class
definitions can be declared in any order, anywhere in a program, but
may not be referenced until after they are declared.

\medskip{}
{\raggedright By convention, the Main form of the IDE displays the
first line of a program as a title of the program. Therefore it is
a good idea (but not required) to start a program with a one line
descriptive comment. Follow that with any global variable declarations,
then any function or class definitions, and finally program statements.\par}

\medskip{}
{\raggedright Semi-colons can be used to separate statements for clarity,
but are not required.\par}


\subsection{Lexical Conventions}

Identifiers in Topaz can be any string of letters, digits, and underscores,
up to sixteen characters long and must begin with a letter. 

\medskip{}
{\raggedright The following identifiers are keywords, and may not
be used otherwise:\par}

\medskip{}
{\centering \begin{tabular}{ccccc}
\texttt{and}&
\texttt{break}&
\texttt{case}&
\texttt{class}&
\texttt{def}\\
\texttt{do}&
\texttt{else}&
\texttt{elsif}&
\texttt{end}&
\texttt{for}\\
\texttt{if}&
\texttt{mod}&
\texttt{not}&
\texttt{or}&
\texttt{private}\\
\texttt{protected}&
\texttt{public}&
\texttt{return}&
\texttt{self}&
\texttt{super}\\
\texttt{then}&
\texttt{unless}&
\texttt{until}&
\texttt{when}&
\texttt{while}\\
\end{tabular}\par}

\medskip{}
{\raggedright The following strings denote other tokens recognized
by the parser:\par}

\begin{lyxcode}
{\centering (~)~{[}~{]}~\{~\}~.~;~,~\&~+~-~{*}~/~||~\&\&~==~=~>~>=~<~<=~<>~!\par}
\end{lyxcode}
\medskip{}
{\raggedright And the following keywords are reserved for future use:\par}

\begin{lyxcode}
{\centering in~include~foreach~module~next~require~static\par}
\end{lyxcode}
\medskip{}
{\raggedright Topaz is a case sensitive language. {}``Case'' and
{}``case'' are two separate identifiers.\par}

\medskip{}
{\raggedright \emph{String Constants} can be up to 64 characters long,
delimited by matching single or double quotes, and can contain the
C-like escape sequences: \par}

\vspace{0.3cm}
{\raggedright \begin{tabular}{rc}
new line:&
\texttt{\textbackslash{}n}\\
carriage return:&
\texttt{\textbackslash{}r}\\
double quote:&
\texttt{\textbackslash{}''}\\
single quote:&
\texttt{\textbackslash{}'}\\
tab:&
\texttt{\textbackslash{}t}\\
backslash:&
\texttt{\textbackslash{}\textbackslash{}}\\
\end{tabular}\par}
\vspace{0.3cm}

{\raggedright Examples of valid string constants are:\par}

\begin{itemize}
\item \texttt{\char`\"{}hello, world\char`\"{} }
\item \texttt{'Programming should be fun' }
\item \texttt{\char`\"{}This \textbackslash{}'word\textbackslash{}' is quoted\char`\"{}}
\end{itemize}
\emph{Numerical Constants -} Standard octal, hexadecimal, and scientific
notation is supported for numerical constants. Examples of valid numerical
constants are:

\begin{itemize}
\item \texttt{42}
\item 0377
\item \texttt{0xffff}
\item \texttt{3.14159 }
\item \texttt{707e-3 }
\item \texttt{1.45E5}
\end{itemize}
\emph{Whitespace} is ignored outside of string constants and consists
of: blanks, tabs, newlines, and carriage returns.

\medskip{}
{\raggedright \emph{Comments} are introduced with a \texttt{\#} and
continue until the next new line character.\par}


\subsection{Data Types}


\subsubsection{Primitive Types}

There are three primitive types built into Topaz: 

\vspace{0.3cm}
{\centering \begin{tabular}{ll}
Integer&
a 32 bit signed integer\\
String&
a single or double quote delimited string not more than 64 characters
long\\
Real&
a 32 bit IEEE single precision float\\
\end{tabular}\par}
\vspace{0.3cm}

{\raggedright In addition to the basic data types, Topaz also supports
Arrays, Hashes, and Objects.\par}


\subsubsection{Arrays}

{\raggedright Arrays are created by assigning an \emph{array initializer}
to a variable. An array initializer is a comma-separated string of
values between square brackets. You can use the special \emph{array
repeat operator} to easily assign multiple values. For example:\par}

\begin{lyxcode}
arr~=~{[}~1,~3.14159,~''some~text'',~7x0~{]}
\end{lyxcode}
will create an array of ten elements -

\begin{lyxcode}
arr{[}0{]}~=~1

arr{[}1{]}~=~3.14159

arr{[}2{]}~=~''some~text''

arr{[}3{]}~=~0



arr{[}9{]}~=~0
\end{lyxcode}
The format of the array repeat operator is: \emph{count} \texttt{x}
\emph{value.} The count must be an integer constant, the {}``\texttt{x}''
must be lowercase, and the value can be any valid expression. It is
only valid inside array initializers. 

\medskip{}
{\raggedright You can also create n dimension arrays by assigning
arrays to array elements. For example:\par}

\begin{lyxcode}
arr~=~{[}~0,~1,~{[}10,20{]}~{]}
\end{lyxcode}
will create an array with three elements, the third being another
array, and containing the following values:

\begin{lyxcode}
arr{[}0{]}~~~~=~0

arr{[}1{]}~~~~=~1

arr{[}2{]}{[}0{]}~=~10

arr{[}2{]}{[}1{]}~=~20
\end{lyxcode}
\medskip{}
{\raggedright Array indexes always start at zero and continue to length-1
. It is a runtime error to try to access arrays with an invalid index.
Memory to hold arrays is allocated from the heap, and can be freed
by simply assigning a zero to the array variable.\par}

\medskip{}
{\raggedright Individual characters of a string variable may be accessed
like an array. For example, after executing the two statements:\par}

\begin{lyxcode}
month~=~''November''

first~=~month{[}0{]}
\end{lyxcode}
the variable \texttt{first} will contain an integer with the value
of 78 (the ascii value of {}``N'').


\subsubsection{Hashes}

{\raggedright Hashes can be created with \emph{hash initializers},
or by simply assigning key/value pairs. A hash initializer is a list
of key/value pairs between braces. Key are separated from values by
the sequence {}``=>''. The key value pairs are separated with commas.
For example, the next two statements:\par}

\begin{lyxcode}
months~=~\{~''Jan''~=>~1,~''Feb''~=>~2~\}

months\{''Mar''\}~=~3
\end{lyxcode}
will first create a hash with two elements, and then add another.
The final result will be -

\begin{lyxcode}
months\{''Jan''\}~=~1

months\{''Feb''\}~=~2

months\{''Mar''\}~=~3
\end{lyxcode}
Memory for hashes is allocated from the heap, and can be freed by
simply assigning a zero to the hash variable.


\subsubsection{Objects}

Objects are 32 bit references to an object instance that are created
by instantiating a class using its \texttt{new} method. 


\subsection{Variables and Constants}

Topaz variables are type-less and are declared by assigning a value
to them with an assignment statement. The \emph{value} of a variable
has a type associated with it.

\begin{itemize}
\item Variables start with a lowercase letter, and are followed by up to
15 other letters, numbers, or underscore characters. 
\item Constants start with an uppercase letter, and are followed by up to
15 other letters, numbers, or underscore characters. Once a constant
is defined, it is a compile-time error to attempt to redefine it.
\end{itemize}

\subsection{Expressions and Operators}


\subsubsection{Expressions}

Expressions combine variables and constants (the operands) with operators
to produce new values. 


\subsubsection{Operators}

\medskip{}
{\raggedright Topaz supports a standard set of operators. The following
table lists all the operators and the types they operate on:\par}

\vspace{0.3cm}
{\centering \begin{tabular}{rccccccl}
 \underbar{}&
&
&
&
&
&
&
\emph{\underbar{Valid operand types}}\\
logical negation:&
\texttt{not}&
\texttt{!}&
&
&
&
&
integer\\
multiplicative:&
\texttt{{*}}&
\texttt{/}&
&
&
&
&
integer or real\\
modulus:&
\texttt{mod}&
\texttt{\%}&
&
&
&
&
integer\\
additive:&
\texttt{+}&
-&
&
&
&
&
integer or real\\
bitwise logical:&
\texttt{and}&
\texttt{\&\&}&
\texttt{or}&
\texttt{||}&
&
&
integer\\
relational:&
\texttt{==}&
\texttt{<>}&
\texttt{<}&
\texttt{<=}&
\texttt{>}&
\texttt{>=}&
integer or real\\
\end{tabular}\par}
\vspace{0.3cm}

\medskip{}
{\raggedright Note that are are two operators with identical functions
for the \emph{logcial negation, modulus}, \emph{bitwise and} and \emph{bitwise
or} operators.The logical negation operators only operate on a single
operand. All the rest operate on two operands. Parenthesis can be
used to group expressions together. \par}


\subsubsection{Operator Precedence}

\medskip{}
{\raggedright Operator precedence is similar to PASCAL where there
are only four levels of operator precedence:\par}

\medskip{}
{\centering \begin{tabular}{ccccccc}
Highest:&
\texttt{not}&
\texttt{!}&
&
&
&
\\
&
\texttt{{*}}&
\texttt{/}&
\texttt{and}&
\texttt{\&\&}&
\texttt{mod}&
\texttt{\%}\\
&
\texttt{+}&
\texttt{-}&
\texttt{or}&
\texttt{||}&
&
\\
Lowest:&
\texttt{==}&
\texttt{<>}&
\texttt{<}&
\texttt{<=}&
\texttt{>}&
\texttt{>=}\\
\end{tabular}\par}

\medskip{}
{\raggedright Operators at the same precedence level are evaluated
from left to right. When in doubt, use parenthesis to group your expressions
together.\par}


\subsubsection{Automatic Type Conversion}

Topaz provides automatic type conversion between values during runtime.
When an arithmetic expression is being evaluated using the relational,
addititive or multiplicative operators, the two operands will automatically
be converted according to the following rules:

\begin{enumerate}
\item If an operand is a string, it will be converted to a real if there
is a decimal point present. Otherwise, it will be converted to an
integer.
\item If both operand types are ints or reals, the resulant type is the
same as the operands.
\item An integer and a real make a real.
\end{enumerate}
For example:

\begin{itemize}
\item 1 + 3.1415 = 4.1415 
\item 1 + \char`\"{}2.0\char`\"{} = 3.0 
\item \char`\"{}1.5\char`\"{} + 2.0 = 3.5
\item ''10'' + ''20'' = 30
\end{itemize}

\subsection{Statements}

\medskip{}
{\raggedright When describing the syntax of Topaz statements, optional
keywords are enclosed by square brackets.\par}


\subsubsection{The assignment statement}

\begin{lyxcode}
\medskip{}
\emph{Lhs}~=~\emph{Expr}
\end{lyxcode}
\medskip{}
{\raggedright The assignment statement is used to declare variables
as well as assign new values to existing variables. The left hand
side can be a new or existing variable. The right hand side can be
any normal expression, array initializer, hash initializer, or function
or method call.\par}


\subsubsection{The \texttt{if} statement}

\medskip{}
{\raggedright The basic format of the \texttt{if} statement is as
follows:\par}

\begin{lyxcode}
\medskip{}
if~\emph{condition}~{[}then{]}

~\emph{~statementlist}

elsif~\emph{condition}~{[}then{]}

~\emph{~statementlist}

else

~\emph{~statementlist}

\medskip{}
end
\end{lyxcode}
\medskip{}
{\raggedright The condition is an expression which controls what the
rest of the statement will do. If condition is true, the the next
block of code will be executed. The \texttt{elsif} \textsf{and} \texttt{else}
blocks are optional. Note however that unlike certain other languages,
the \texttt{end} keyword is required.\par}


\subsubsection{The \texttt{unless} statement}

\begin{lyxcode}
\medskip{}
unless~\emph{condition}~{[}then{]}~

~~\emph{statementlist~}

else

~~\emph{statementlist}

end
\end{lyxcode}
\medskip{}
{\raggedright The \texttt{unless} statement is similar the the \texttt{if}
statement, except that the next block of code will be executed when
when the condition is false. The \texttt{else} block is optional.\par}


\subsubsection{The \texttt{case} statement}

\begin{lyxcode}
\medskip{}
case~\emph{expression}

when~\emph{condition}~{[}then{]}

~~\emph{statementlist}

else

~~\emph{statementlist}

end
\end{lyxcode}
\medskip{}
{\raggedright The \texttt{case} statement is yet another form of the
\texttt{if} statement. The expression after the case keyword is evaluated.
If the result matches the value of the expression in one of the \texttt{when}
or \texttt{else} statements, then that block of code will be executed.
There can be zero or more \texttt{when} blocks and zero or one \texttt{else}
blocks.\par}


\subsubsection{The \texttt{for} statement}

\begin{lyxcode}
\medskip{}
for(~\emph{init};~\emph{condition};~\emph{incr}~)

~~\emph{statementlist}~

end
\end{lyxcode}
\medskip{}
{\raggedright The \texttt{for} statement is very similar its C language
counterpart. The \emph{init} part is an assignment statement which
is executed at the beginning of the loop. Then \emph{condition} is
evaluated, and while it is true, the \emph{statementlist} will be
executed. At the end of the loop, \emph{incr} is evaluated, and the
loop starts over. All three parts (init,condition,incr) are optional.\par}


\subsubsection{The \texttt{while} statement}

\begin{lyxcode}
\medskip{}
while~\emph{condition}~{[}do{]}~

~~\emph{statementlist}

end
\end{lyxcode}
\medskip{}
{\raggedright The \texttt{while} statement is the simplest of loops.
The condition part is evaluated at the beginning of each loop, and
while true, the statement list will be executed. \par}


\subsubsection{The \texttt{until} statement}

\begin{lyxcode}
\medskip{}
until~\emph{condition}~{[}do{]}

~~\emph{statementlist}

end
\end{lyxcode}
\medskip{}
{\raggedright The \texttt{until} statement is the opposite of the
while. The condition part is evaluated at the beginning of each loop,
and while false, the statement list will be executed. \par}


\subsubsection{The \texttt{break} statement}

\begin{lyxcode}
\medskip{}
break
\end{lyxcode}
\medskip{}
{\raggedright The \texttt{break} statement is used to break out of
\texttt{for}, \texttt{while}, and \texttt{until} loops. It transfers
control to the next statement after the loop.\par}


\subsubsection{The \texttt{return} statement}

\begin{lyxcode}
\medskip{}
return~\emph{expression}
\end{lyxcode}
\medskip{}
{\raggedright The \texttt{return} statement is used to return from
functions and class methods. \emph{expression} is the optional return
value.\par}


\subsubsection{The \texttt{public} statement}

\begin{lyxcode}
\medskip{}
public
\end{lyxcode}
\medskip{}
{\raggedright The \texttt{publi}c statement is only valid within class
definitions and signals that any following fields or methods will
be publicly available.\par}


\subsubsection{The \texttt{protected} statement}

\begin{lyxcode}
\medskip{}
protected
\end{lyxcode}
\medskip{}
{\raggedright The \texttt{protected} statement is only valid within
class definitions and signals that any following fields or methods
will be protected. That is, they are only visible within the class
they are defined and to sublcasses.\par}


\subsubsection{The \texttt{private} statement}

\begin{lyxcode}
\medskip{}
private
\end{lyxcode}
\medskip{}
{\raggedright The \texttt{private} statement is only valid within
class definitions and signals that any following fields or methods
will be private. That is, they are only visible within the class they
are defined in.\par}


\subsection{Functions}

\medskip{}
{\raggedright Functions in Topaz are similar to functions in most
other procedural programming languages. Their basic syntax is:\par}

\begin{lyxcode}
def~\emph{functionname}(\emph{param}1,param2,)

~~\emph{statementlist}

end
\end{lyxcode}
The params enclosed by parenthesis are optional. The \texttt{return}
statement can optionally be used to return a value.

\medskip{}
{\raggedright Functions can be defined anywhere in a program, but
can only be referenced after they are defined. Any new variables declared
within a function are considered local variables and are only visible
within that function. Any variables, functions, or classes declared
in the global scope before the current function definition will also
be visible.\par}


\subsection{Classes}

\medskip{}
{\raggedright The basic sytax of a class definition is:\par}

\begin{lyxcode}
class~\emph{Name}~<~\emph{Superclass}

~~\emph{body}

end
\end{lyxcode}
The class name must begin with an uppcase letter. The sequence \texttt{<}
\texttt{\emph{Superclass}} is optional. A class body consists of a
series of one or more field and method definitions. By default, everything
in a class has public scope. The \texttt{public}, \textsf{}\texttt{protected},
and \texttt{private} statements can be used the change the visibility
of following definitions.


\subsubsection{Instantiation}

Every class implicitly has a \texttt{new} method which will automatically
create a new object and initialize its fields. For example:

\begin{lyxcode}
point~=~Point3D.new
\end{lyxcode}
will create a new object of class Point3D and assign a reference of
it to the variable point.


\subsubsection{Fields}

Class fields must be initialized with static values - either int,
real, string, hash or array. You cannot intialize a field with an
object.


\subsubsection{Methods}

Class methods are simply functions declared within a class definition.
The syntax is exactly the same. A method with the name \texttt{init}
is considered a class constructor. See the section on class constructors.


\subsubsection{Scope}

{\raggedright Class fields an methods can be at one of three scope
levels: \par}

\begin{lyxlist}{00.00.0000}
\item [Public]- visible anywhere within it's class after its declaration,
and in any instantiated objects.
\item [Protected]- visible within the class it is defined in, and any subclasses
\item [Private]- visible only within the class it is defined in
\end{lyxlist}
By default, everything in a class is public. You can change the scope
using the public, protected, or private keywords.

\medskip{}
{\raggedright Class fields and method declarations only belong to
the class they are defined in, therefore they can and will override
any previous identifiers.\par}


\subsubsection{Constructors}

{\raggedright Class fields are automatically initialized when a classes
\texttt{new} method is called. In addition, you may create a user
defined class initializer method called \texttt{init}. If the \texttt{init}
method declaration does not have any formal parameters, it will be
automatically called right after the fields are initialized. If it
does have parameters, you can pass them via the \texttt{new} call.
If it does have parameters and you do not pass them via the new call,
the \texttt{init} method will not automatically be called, however
you may explicitly call it yourself (with the proper number of parameters).
When a class is instantiated all of its superclasses fields will be
initialized, however you must explicitly call any \texttt{init} methods.
Generally you would call a superclasses \texttt{init} method from
the base classes \texttt{init} method. For example:\par}

\begin{lyxcode}
class~Point2D

~~x=0;~y=0

~~def~init(x,y)

~~~self.x=x;~self.y=y

~~end

end

class~Point3D~<~Point2D

~~z=0

~~def~init(xx,yy,zz)

~~~~super.init(xx,yy)~~~~~~\#~or~-~x=xx;~y=yy

~~~~z=zz

~~end

end

p1~=~Point2D.new~~~~~~~~~~~\#~p1.x=0,~p2.y=0

p2~=~Point2D.new(10,10)~~~~\#~p2.x=10,~p2.y=10

p3~=~Point3D.new~~~~~~~~~~~\#~p3.x=0,~p3.y=0,~p3.z=0

p4~=~Point3D.new(10,10,10)~\#~p4.x=10,~p4.y=10,~p4.z=10
\end{lyxcode}

\subsubsection{Inheritance}

\medskip{}
{\raggedright Topaz supports single inheritance with chained subclassing.
That is - a class can have only one superclass, however that class
can have a superclass, and so on.\par}


\subsection{Scope}

The scope rules in Topaz are pretty simple. 

\begin{itemize}
\item The built-in constants, functions, and classes are always accessible
anywhere in a program 
\item A user defined constant, variable, function, or class is not accessible
until after it is declared.
\item All function and class names are global as well as any constants or
variables declared outside of function or class definitions. 
\item Any global constants, variables, functions, or classes will be accessible
within any subsequent function or class definitions.
\item If a constant or variable is declared for the first time within a
function or class method, it is considered local and will only be
visible within that function or method.
\item Class fields and methods are alway local to the class they are defined
in and can override global identifiers.
\end{itemize}

\subsection{Memory Management}

\medskip{}
{\raggedright Topaz automatically allocates and frees memory to hold
values as needed. Array, Hashes, and Objects are all allocated from
the PalmOS 64k heap area when they are declared. Caution is recommended
when creating many objects or large arrays and hashes. Any memory
allocated inside of a method is automatically freed (except for any
return value) when returning from the method. If you allocate a large
array or hash, and would later like to free the memory associated
with it, simply assign a zero to the variable.\par}


\section{Standard Library Reference}


\subsection{Constants}

\texttt{True }

The integer one.

\medskip{}
{\raggedright \texttt{False}\par}

The integer zero.


\subsection{Functions}

Functions are described with the convention: return\_type=function(typeName),
where type can be {}``i'' for integer, {}``r'' for real, {}``s''
for string, or {}``e'' for expression.


\subsubsection{Types}

\texttt{s=getType(e)}

Returns: {}``int'', {}``real'', {}``string'', {}``array'',
{}``hash'', or {}``object''.

\medskip{}
{\raggedright \texttt{v=setType(e)}\par}

Returns: e converted to type sType; where sType can be - {}``int'',
{}``real'', or {}``string''.

\medskip{}
{\raggedright \texttt{i=isInt(e)}\par}

Returns: True if var is an int, False otherwise.

\medskip{}
{\raggedright \texttt{i=toInt(e)}\par}

Returns: e converted to an integer.

\medskip{}
{\raggedright \texttt{i=isReal(e)}\par}

Returns: True if var is a real, False otherwise.

\medskip{}
{\raggedright \texttt{r=toReal(e)}\par}

Returns: e converted to a real.

\medskip{}
{\raggedright \texttt{i=isStr(e)}\par}

Returns: True if var is a string, False otherwise.

\medskip{}
{\raggedright \texttt{s=toString(e)}\par}

Returns: e converted to a string.


\subsubsection{Math}

\texttt{r=sin(rDeg) }

The sine of \emph{x} in degrees.

\medskip{}
{\raggedright \texttt{r=cos(rDeg)}\par}

The cosine of \emph{x} in degrees.

\medskip{}
{\raggedright \texttt{r=tan(rDeg)}\par}

The tangent of \emph{x} in degrees.

\medskip{}
{\raggedright \texttt{r=sqrt(rDeg) }\par}

The square root of \emph{x.}

\medskip{}
{\raggedright \texttt{r=abs(rDeg)}\par}

The absolute value of \emph{x.}


\subsubsection{Graphics and Events}

\texttt{i=event(iTimeout)}

Waits iTimeout ticks for an event. Events returned are:

0 - nilEvent

1 - penDown

2 - penMove

3 - penUp

4 - keyDown

3000 or greater - Control ID after control select event.

\medskip{}
{\raggedright \texttt{yield}\par}

Gives the OS a chance to process events. Use inside of loops to prevent
lockups.

\medskip{}
{\raggedright \texttt{i=penx}\par}

Returns: the last x coordinate of the stylus.

\medskip{}
{\raggedright \texttt{i=peny}\par}

Returns: the last y coordinate of the stylus.

\medskip{}
{\raggedright \texttt{i=key}\par}

Returns: the code of the last Graffiti character entered.

\medskip{}
{\raggedright \texttt{clear}\par}

Clears the entire screen.

\medskip{}
{\raggedright \texttt{drawpoint(iX,iY)}\par}

Draws a pixel at X,Y.

\medskip{}
{\raggedright \texttt{drawline(iX1,iY1,iX2,iY2)}\par}

Draws a line from X1,Y1 to X2,Y2.

\medskip{}
{\raggedright \texttt{drawrect(iTopLeftX,iTopLeftY,iExtentX,iExtentY)}\par}

Draws a rectangle from TopLeftX,TopLeftY with the sides length ExtentX
and ExtentY.

\medskip{}
{\raggedright \texttt{title(sTitle)}\par}

Draws a PalmOS style title bar at the top of the screen.


\subsubsection{Miscellaneous}

\texttt{print(v\{,v\})}

{\raggedright \texttt{println(v\{,v\})}\par}

Prints a comma separated list of values to the console using default
formatting. Println appends a newline character, print does not.

\medskip{}
{\raggedright \texttt{printf(sFormat,v\{,v\})}\par}

{\raggedright \texttt{printxy(iX,iy,sFormat,v,\{,v\})}\par}

Works just like the standard C printf. Printf formats and prints to
the next line on the console, printxy prints starting at pixel location
X,Y. Supports all of the standard width and precision modifiers and
the following format codes:

\texttt{\%d - decimal }

\texttt{\%s - string}

\texttt{\%c - character}

\texttt{\%o - octal}

\texttt{\%x - lower case hex}

\texttt{\%X - upper case hex}

\texttt{\%f - real}

\texttt{\%e - lower case exponential notation}

\texttt{\%E - upper case exponential notation}

\texttt{\%i - integer}

\texttt{\%b - binary}

\medskip{}
{\raggedright \texttt{i=len(e)}\par}

Returns the length of e. If e is an array or hash len will return
the number of elements. For a string, it will return the length of
the string. For int's and real's, it returns one.

\medskip{}
{\raggedright \texttt{beep}\par}

Triggers the Palm's built-in piezo element.

\medskip{}
{\raggedright \texttt{delay(iTicks)}\par}

Waits for a specified amount of time.

\medskip{}
{\raggedright \texttt{i=random(i)}\par}

Returns: a random number between 0 and i.

\medskip{}
{\raggedright \texttt{dup(v)}\par}

Creates a copy of the argument.


\subsection{Classes}


\subsubsection{Misc}

\medskip{}
{\raggedright \texttt{Object}\par}


\subsubsection{User Interface}

\texttt{Fld}

\medskip{}
{\raggedright \texttt{Btn}\par}

\medskip{}
{\raggedright \texttt{PBtn}\par}

\medskip{}
{\raggedright \texttt{RBtn}\par}

\medskip{}
{\raggedright \texttt{CBox}\par}

\medskip{}
{\raggedright \texttt{Lst}\par}

\medskip{}
{\raggedright \texttt{Trg}\par}

\medskip{}
\raggedright \texttt{Sel}
\end{document}
